% !TeX spellcheck = ru_RU
% !TEX root = vkr.tex

\section{Обзор}
\label{sec:relatedworks}
В данном разделе будут проанализированны веб-сервисы, связанные с отловом домашних животных,
а также рассмотрены технологии для реализации клиентской части веб-сервиса.


\subsection{Обзор существующих решений}
\subsubsection{Авито}
Минусы:
  \begin{itemize}
      \item Многофункциональнсоть платформы и сложность поиска: Авито не специализируется на конкретной услуге,
      поиск по запросу "отлов животных" выдаст объявления об отлове, преложение аренды-продажи оборудования,
      объявления о передаче животных в добрые руки и тд., из за чего поиск подходящей услуги может занять время
  \end{itemize}
\quad Плюсы:
  \begin{itemize}
      \item Отзывы и рейтинги: На «Авито» пользователи могут оставлять отзывы о работе отловщиков, что
      позволяет другим пользователям принимать более обоснованное решение при выборе исполнителя. Это
        помогает избежать сотрудничества с некомпетентными или недобросовестными специалистами.
      \item Комментирование и обсуждение: Некоторые объявления могут также содержать комментарии или вопросы
      от других пользователей, что позволяет получить дополнительную информацию о специалистах и их услугах.
      \item Широкая аудитория. Большое количество предложений.
  \end{itemize}
\subsubsection{Вк}
  Минусы:
  \begin{itemize}
      \item Cложность поиска: Отсутствие единой функциональности обеспечивающей поиск улуг отловщиков. Все объявления
      находятся в множестве груп и чатах, в каждом из которых придется искать информацию о требуемой услуге.
  \end{itemize}
\quad Плюсы:
  \begin{itemize}
      \item Наличие сообществ с волонтерами, готовыми оказать помощь.
      \item Дружелюбное комьюнити.
  \end{itemize}
\subsubsection{Профи.ру}
  Минусы:
  \begin{itemize}
    \item Строгий поиск. Необходимость знать всю информацию о желаемой улуге заранее.
    \item Нет возможности предварительного просмотра специалистов и услуг без регистрации и оставления заявки.
  \end{itemize}



\subsection{Обзор используемых технологий}
Для работы над frontend частью веб-сервиса был выбран фреймворк angular~\cite{angularLanguage} за его
модульность и компонентный подход, а также поддержку typescript~\cite{typescriptLanguage} с строгой типизацией,
что повышает простоту разработки в команде, мастшабируемость и читаемость.
В следствии стремления повышения модульности кода был выбран CCSS~\cite{ccssLanguage} вместо CSS.

\subsection{Выводы}

Существует множество многопрофильных платформ, не специализирующихся на отлове и помощи бездомным животным, но предоставляющих соотвествующие услуги.
В связи с этим поиск подходящего человека или оборудования может занимать лишнее время из за отсутсвия удобной системы поиска и фильтров.
Поэтому необходима единая платформа заточенная на помощь бездомным животным, в которой будет реализованна вся необходимая для этогофункциональность, в частности, раздел для поиска отловщиков или оборудования для отлова.
